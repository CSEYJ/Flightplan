\documentclass{article}

  \usepackage{booktabs}

\begin{document}

  \title{FEC booster packet format}

  \author{Hans Giesen}

  \maketitle  

  \begin{table}
    \begin{tabular}{l l}
      \toprule
        Data size & Description \\
      \midrule
        6 octets & Destination MAC address \\
        6 octets & Source MAC address \\
        2 octets & EtherType of FEC booster (0x081C) \\
        3 bits & Traffic class: \\
               & 000: $k = 5$, $h = 1$ \\
               & 001: $k = 50$, $h = 1$ \\
               & 010: $k = 50$, $h = 5$ \\
        5 bits & Block number \\
        1 octet & Frame number ($< k$) \\
        2 octets & Original EtherType \\
        ? octets & Original payload \\
      \bottomrule
    \end{tabular}
    \caption{Data frame format}
  \end{table}

  \begin{table}
    \begin{tabular}{l l}
      \toprule
        Data size & Description \\
      \midrule
        6 octets & Destination MAC address \\
        6 octets & Source MAC address \\
        2 octets & EtherType of FEC booster (0x081C) \\
        3 bits & Traffic class: \\
               & 00: $k = 5$, $h = 1$ \\
               & 01: $k = 50$, $h = 1$ \\
               & 02: $k = 50$, $h = 5$ \\
        5 bits & Block number \\
        1 octet & Frame number ($\ge k$) \\
        2 octets & Original EtherType \\
	2 bits & Reserved \\
        14 bits & Payload size in octets (size of original Ethernet frame) \\
        ? octets & Encoded payload (entire Ethernet frame) \\
      \bottomrule
    \end{tabular}
    \caption{Parity frame format}
  \end{table}

\end{document}
