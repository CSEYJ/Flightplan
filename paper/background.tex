\section{Background}
A communication network is architected to handle variation in its state
arising from environmental interference, component failure, congestion and
corruption of data. In this paper we describe a technique to enhance the
reliability of networks with failing links.

Our approach could be compared to existing link-layer mitigations for
unreliable transmission, such as network-wide hop-by-hop protection, as in
X.25~\cite{X25}, and link-layer retransmission as in the 802.11~\cite{WiFi} family of
standards. The mitigation chosen for each system mostly depends on
the transmission medium: X.25 was designed to work with unreliable links,
whereas 802.11 uses a shared medium. Our thinking is similar,
and our design is based on properties of the medium: we concentrate on wireline
10Gbps Ethernet; typically such links are reliable (i.e., have low error
rates) and they are not shared (i.e., point-to-point).
Instead of using an ARQ scheme as in X.25 and 802.11, we use forward
error-correction (FEC).
This simplifies our design, obviating the need for retransmission windows and this
diminishes the memory needed for in-flight data since the sender will
not attempt to resend a frame.

Our link-layer FEC complements the physical-layer FEC that is used in
high-capacity Ethernet links: the physical-layer FEC helps the link sustain a
given capacity over longer distances, whereas our FEC is intended to mitigate
errors that do not arise because of challenging environmental factors; rather
the errors arise because of physical damage to the link or failing transceivers~\cite{Zhuo:2017:UMP:3098822.3098849}.

Our approach also complements higher-layer reliability measures, as provided by
TCP for example.  TCP provides end-to-end reliability, whereas we concentrate
on link-level reliability. Unlike TCP, we are able to distinguish congestion
from corruption as the cause of packet loss, and we are able to locate the
lossy links in the network. Thus we can react to them much more quickly than
TCP at the end-points; we evaluate this in~\S\ref{sec:evaluation}. As with TCP, our mechanism results in a reduced
transmission rate, but this is necessary since the link capacity has been
reduce because of the link's failure.

In this paper we focus on Clos
topologies~\cite{clos_bstj1953}, 
%% surely we can give Clos credit for Clos networks... -- AMD
which are used in datacenters~\cite{Singh:2016:JRD:2991470.2975159}.
We believe that our design could be useful in mitigating faulty links in a Clos
topology due to the large number links it uses when compared to a
hierarchical topology~\cite{Al-Fares:2008:SCD:1402946.1402967}, which 
%makes  -- AMD don't want "makes exposes" -- pick one...
exposes it more to link-related faults.

Centralized approaches have been described in the literature to mitigate
faulty links in a WAN~\cite{traffic-engineering-with-forward-fault-correction}
and datacenters~\cite{Zhuo:2017:UMP:3098822.3098849}. In comparison our approach
is not centralized, and takes place in the network: a switch can activate
FEC with adjacent switches over faulty links until the links are replaced or
repaired.
