\documentclass[sigconf]{acmart}

\usepackage{booktabs} % For formal tables

% Copyright
%\setcopyright{none}
%\setcopyright{acmcopyright}
%\setcopyright{acmlicensed}
\setcopyright{rightsretained}
%\setcopyright{usgov}
%\setcopyright{usgovmixed}
%\setcopyright{cagov}
%\setcopyright{cagovmixed}


% DOI
\acmDOI{10.475/123_4}

% ISBN
\acmISBN{123-4567-24-567/08/06}

%Conference
\acmConference[SHORTNAME'17]{ACM Long Conference Name conference}{July 1997}{City, State, Country} 
\acmYear{2017}
\copyrightyear{2017}

\acmPrice{15.00}


\begin{document}
\title{Tango: Dynamic Booster Coupling for Data Center Networks}
%\title{SIG Proceedings Paper in LaTeX Format}
%\titlenote{Produces the permission block, and copyright information}
%\subtitle{Extended Abstract}

\author{Penn-DCOMP}
%\author{Firstname Lastname}
%\authornote{Note}
%\orcid{1234-5678-9012}
%\affiliation{%
%  \institution{Affiliation}
%  \streetaddress{Address}
%  \city{City} 
%  \state{State} 
%  \postcode{Zipcode}
%}
%\email{email@domain.com}
%
%\author{Firstname Lastname}
%\orcid{1234-5678-9012}
%\affiliation{%
%  \institution{Affiliation}
%  \streetaddress{Address}
%  \city{City} 
%  \state{State} 
%  \postcode{Zipcode}
%}
%\email{email@domain.com}
%
%\author{Firstname Lastname}
%\orcid{1234-5678-9012}
%\affiliation{%
%  \institution{Affiliation}
%}
%\email{email@domain.com}
%
%\author{Firstname Lastname}
%\orcid{1234-5678-9012}
%\affiliation{%
%  \institution{Affiliation}
%}
%\email{email@domain.com}
%
%\author{Firstname Lastname}
%\orcid{1234-5678-9012}
%\affiliation{%
%  \institution{Affiliation}
%}
%\email{email@domain.com}


% The default list of authors is too long for headers}
\renewcommand{\shortauthors}{F. Lastname et al.}


\begin{abstract}
Applications running on a data center network typically expect good performance
from a network whose use they compete for in an uncoordinated manner, but
meeting such applications' needs in a high-performance environment is a
persistent challenge. This is exacerbated in the presence of rolling failures
within hosts and networks, which is a necessary feature of networks at data
center scale.

Research on data center networking has sought to provide the means to
express and execute performance-related preferences to better serve
applications by better utilising the network.

(Something about the fit of current techniques, depending on the set of use-cases we ultimately settle on.)

In this work we describe Tango, a distributed framework comprising network
boosters that can be dynamically activated and chained to improve the network's
performance to better serve applications. Boosters are coupled close to the two
communicating endpoints in the data center, and the intermediate network is
oblivious to the boosting. The boosters can execute in software or on
reconfigurable hardware, depending on a host's capabilities and the
application's priorities.

(Something about why this is truly excellent wrt the state of the art, and what performance objectives this allows us to realise.)
\end{abstract}

%
% The code below should be generated by the tool at
% http://dl.acm.org/ccs.cfm
% Please copy and paste the code instead of the example below. 
%
\begin{CCSXML}
<ccs2012>
 <concept>
  <concept_id>10010520.10010553.10010562</concept_id>
  <concept_desc>Computer systems organization~Embedded systems</concept_desc>
  <concept_significance>500</concept_significance>
 </concept>
 <concept>
  <concept_id>10010520.10010575.10010755</concept_id>
  <concept_desc>Computer systems organization~Redundancy</concept_desc>
  <concept_significance>300</concept_significance>
 </concept>
 <concept>
  <concept_id>10010520.10010553.10010554</concept_id>
  <concept_desc>Computer systems organization~Robotics</concept_desc>
  <concept_significance>100</concept_significance>
 </concept>
 <concept>
  <concept_id>10003033.10003083.10003095</concept_id>
  <concept_desc>Networks~Network reliability</concept_desc>
  <concept_significance>100</concept_significance>
 </concept>
</ccs2012>  
\end{CCSXML}

\ccsdesc[500]{Computer systems organization~Embedded systems}
\ccsdesc[300]{Computer systems organization~Redundancy}
\ccsdesc{Computer systems organization~Robotics}
\ccsdesc[100]{Networks~Network reliability}

% We no longer use \terms command
%\terms{Theory}

\keywords{ACM proceedings}


\maketitle

\section{Introduction}
\section{Mishaps in datacenter networks}
Congestion,
Corruption,
Node failure, etc.
Scheduler error?
\section{Mitigating network mishaps}
On the hosts,
In the network.

Control process to determine when to activate boosters, and possibly what parameters to use.
\section{Use-cases}
\subsection{Forward Error Correction}
Need to specify topology early on.
\subsection{TBD}
\section{Implementation}
Some bits done in P4, and other bits that can co-/entirely run on FPGA or a host's CPU.
\section{Evaluation}
\section{Related work}
Data centers:
Eden.
CorrOpt.
CONGA.

Boosting elsewhere: JMS' work from the 90's.
\section{Conclusion}

\bibliographystyle{ACM-Reference-Format}
\bibliography{sigproc} 

\end{document}
