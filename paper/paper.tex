\documentclass[sigconf]{acmart}

\newcommand{\FIXME}[1]{{\color{red}{\textbf{FIXME:}#1}}}

\usepackage{booktabs} % For formal tables

% Copyright
%\setcopyright{none}
%\setcopyright{acmcopyright}
%\setcopyright{acmlicensed}
\setcopyright{rightsretained}
%\setcopyright{usgov}
%\setcopyright{usgovmixed}
%\setcopyright{cagov}
%\setcopyright{cagovmixed}


% DOI
\acmDOI{10.475/123_4}

% ISBN
\acmISBN{123-4567-24-567/08/06}

%Conference
\acmConference[SHORTNAME'17]{ACM Long Conference Name conference}{July 1997}{City, State, Country} 
\acmYear{2017}
\copyrightyear{2017}

\acmPrice{15.00}

\newcommand{\OurSys}{Wharf\xspace}

\begin{document}
\title{Mitigating faulty links using in-network computing}
%\title{SIG Proceedings Paper in LaTeX Format}
%\titlenote{Produces the permission block, and copyright information}
%\subtitle{Extended Abstract}

\author{Anirudh Chelluri \kern1em
 Andr\'e DeHon \kern1em
 Hans Giesen \kern1em
 Boon Thau Loo \kern1em
 Nishanth Prabhu \kern1em
 Lei Shi \kern1em
 John Sonchack \kern1em
 Nik Sultana}
\affiliation{%
\institution{University of Pennsylvania}}

%\author{Firstname Lastname}
%\authornote{Note}
%\orcid{1234-5678-9012}
%\affiliation{%
%  \institution{Affiliation}
%  \streetaddress{Address}
%  \city{City} 
%  \state{State} 
%  \postcode{Zipcode}
%}
%\email{email@domain.com}
%
%\author{Firstname Lastname}
%\orcid{1234-5678-9012}
%\affiliation{%
%  \institution{Affiliation}
%  \streetaddress{Address}
%  \city{City} 
%  \state{State} 
%  \postcode{Zipcode}
%}
%\email{email@domain.com}
%
%\author{Firstname Lastname}
%\orcid{1234-5678-9012}
%\affiliation{%
%  \institution{Affiliation}
%}
%\email{email@domain.com}
%
%\author{Firstname Lastname}
%\orcid{1234-5678-9012}
%\affiliation{%
%  \institution{Affiliation}
%}
%\email{email@domain.com}
%
%\author{Firstname Lastname}
%\orcid{1234-5678-9012}
%\affiliation{%
%  \institution{Affiliation}
%}
%\email{email@domain.com}


% The default list of authors is too long for headers}
%\renewcommand{\shortauthors}{F. Lastname et al.}
\renewcommand{\shortauthors}{A. Chelluri et al.}

\begin{abstract}
Failing network links are usually disabled, and packets are routed around them
until the links are repaired.  This is done despite it often being possible
to utilise some of a failing link's capacity. Losing what remains of a link's
capacity is deemed preferable to the erratic effect that unreliable links can
have on application-level behaviour.

We describe a new network function that relies on in-network computing to limit
the erratic effect of failing network links, to enable the continued use of
those links until they can be repaired. We argue that such a network function
can help mitigate rolling failures in datacenter networks, and that our design
can interoperate with existing network architecture and configuration choices,
such as for multi-path routing.

We evaluate our design by using an ns-3 model, and evalaute our implementation on a physical test-bed that includes programmable switches and reconfigurable hardware. We find that we can reliably improve \FIXME{parameter} by \FIXME{proportion}.
\end{abstract}

%
% The code below should be generated by the tool at
% http://dl.acm.org/ccs.cfm
% Please copy and paste the code instead of the example below. 
%
\begin{CCSXML}
<ccs2012>
 <concept>
  <concept_id>10010520.10010553.10010562</concept_id>
  <concept_desc>Computer systems organization~Embedded systems</concept_desc>
  <concept_significance>500</concept_significance>
 </concept>
 <concept>
  <concept_id>10010520.10010575.10010755</concept_id>
  <concept_desc>Computer systems organization~Redundancy</concept_desc>
  <concept_significance>300</concept_significance>
 </concept>
 <concept>
  <concept_id>10010520.10010553.10010554</concept_id>
  <concept_desc>Computer systems organization~Robotics</concept_desc>
  <concept_significance>100</concept_significance>
 </concept>
 <concept>
  <concept_id>10003033.10003083.10003095</concept_id>
  <concept_desc>Networks~Network reliability</concept_desc>
  <concept_significance>100</concept_significance>
 </concept>
</ccs2012>  
\end{CCSXML}

\ccsdesc[500]{Computer systems organization~Embedded systems}
\ccsdesc[300]{Computer systems organization~Redundancy}
\ccsdesc{Computer systems organization~Robotics}
\ccsdesc[100]{Networks~Network reliability}

% We no longer use \terms command
%\terms{Theory}

\keywords{ACM proceedings}


\maketitle

\section{Introduction}
\section{Background}
Fat-tree topology, and port density.
CorrOpt~\cite{Zhuo:2017:UMP:3098822.3098849}.

\section{Design}
\section{Implementation}
\section{Evaluation}
\section{Conclusion}

\bibliographystyle{ACM-Reference-Format}
\bibliography{paper}

\end{document}
